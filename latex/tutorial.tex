\documentclass[a4,12pt]{article}
\usepackage[utf8]{inputenc}
\usepackage[spanish]{babel}
\usepackage[margin=3cm]{geometry}
\usepackage{graphicx}
\usepackage{import}
\usepackage{color}
%\usepackage{times}

\renewcommand{\familydefault}{\sfdefault}


\usepackage{hyperref}

\hypersetup{
    pdfborder = {0 0 0}
}

\title{Trabajo para el departamento de DIS \newline
\LaTeX, GIT y Octave}

\author{Óscar Berrocal Fráhija}



\begin{document}

\maketitle



\begin{abstract}
Este documento tratará de desarrollar un pequeño trabajo en GNU Octave, documentándolo en latex y usando git como repositorio.
\end{abstract}

\tableofcontents

\newpage

\section{Introducción}

Aquí debería explicar de qué va la cosa.

\section{Latex es raro}
Para instalarlo, he tenido que hacer lo siguiente:\newline

>sudo apt-get install texlive-full\newline
Y luego, he tenido que ejecutar sudo texhash para que actualice sus referencias a libs, creo.\newline
Cada vez que quiero generar el PDF, no debo olvidar compilarlo dos veces para que a la segunda ya pille las referencias.


\section{Desarrollo}

Explicaremos todo aquí.

\subsection{Sub1}

Adios.

\subsection{Sub2}

Hola.


%\bibliographystyle{plain}
%\bibliography{referencias}

\end{document}

